\documentclass[a4paper,10pt]{article}

%
%%% INPUT, FONTS, LAYOUT
%

\usepackage{fourier} % change this to change the font

\usepackage[english]{babel}   % Spacing and hyphenation rules for English.
\usepackage[T1]{fontenc} % ensures that ANSI characters <, >, | render correctly and are searchable

\usepackage[
  margin=15mm, 
  top=18ex,
  bottom=10ex, 
  footskip=5ex
]{geometry}


\usepackage{fancyhdr}
\pagestyle{fancy}
\fancyhead[L]{{\sffamily Feedlot working group}}
\fancyfoot[C]{}
\fancyfoot[R]{\thesection-\thepage}
\renewcommand \footrulewidth {2pt}

\usepackage{setspace} % provides \onehalfspacing, \doublespacing
\usepackage{calc} % provides length arithmetic
\usepackage{xcolor} % \color

  

%
%%% CROSS REFERENCING %%%%%%%%%%%%%%%%%%%%%%
%

\usepackage[colorlinks=true,
                linkcolor=teal,
                citecolor=olive,
                destlabel=true,
                bookmarks=true]{hyperref}

% Turns cross-references into links by redefining \refstepcounter
% Turns citations into links by redefining \bibcite
% Turns amsmath equation numbers into links by unknown wizardry
% Provides \href{}{}

\usepackage{bookmark} 
% patches hyperref to simplify processing (no .out file is generated and 
% only needs 2 runs).

%
%%% BIBLIOGRAPHY %%%%%%%%%%%%%%%%%%%%%%%%%%


\usepackage[style=authoryear]{biblatex}
\setcounter{biburllcpenalty}{1}   % freely break URLs at lowercase

% use \addbibresource to register .bib databases

\addbibresource{master.bib}

% About backends:
% bibtex reads both the .aux file and .bib file.
% biber reads ONLY the .bcf file which is generated by biblatex 
% when backend=biber.
%
% About biblatex:
% loads etoolbox
% Use \nopp to suppress p. pp. in postnote. 
% Alternatively, change pagination field in bib entry.
% For arXiv compatibility require versions:
%  .bbl format 2.8 biblatex 3.7 Biber 2.7
%  (as of May 2020, arXiv uses TeXLive 2016)

\usepackage{csquotes} % prevents a warning for using biblatex with babel



\DeclareBibliographyDriver{eprint}{%
  \usebibmacro{bibindex}%
  \usebibmacro{begentry}%
  \usebibmacro{author/editor+others/translator+others}%
  \setunit{\printdelim{nametitledelim}}\newblock
  \usebibmacro{title}%
  \newunit
  \printlist{language}%
  \newunit\newblock
  \usebibmacro{byauthor}%
  \newunit\newblock
  \usebibmacro{byeditor+others}%
  \newunit\newblock
  \printfield{howpublished}%
  \newunit\newblock
  \printfield{type}%
  \newunit
  \printfield{version}%
  \newunit
  \printfield{note}%
  \newunit\newblock
  \usebibmacro{organization+location+date}%
  \newunit\newblock
  \usebibmacro{eprint}%
  \newunit\newblock
  \usebibmacro{addendum+pubstate}%
  \setunit{\bibpagerefpunct}\newblock
  \usebibmacro{pageref}%
  \newunit\newblock
  \iftoggle{bbx:related}
    {\usebibmacro{related:init}%
     \usebibmacro{related}}
    {}%
  \usebibmacro{finentry}}


%%% MATH MACROS %%%%%%%%%%%%%%%%%%%%%%%%%%%%%

\usepackage{calc}
\usepackage{xcolor}
\newcommand \Vol {\mathrm{vol}}



%
%%% TITLE & SPLASH
%

\usepackage{titling}

\pretitle{
  \begin{flushleft} \Huge \sffamily \rule{\textwidth}{0.1ex}\vspace{1ex}}
\posttitle{ \rule{\textwidth}{0.1ex} \end{flushleft}}
\title{ Feedlot: trading with safe, low latency price oracles  \thanks{This research was funded by the Gnosis DAO.} }

\preauthor{\begin{flushleft} \LARGE \sffamily \itshape}
\author{The Feedlot working group}
\postauthor{\end{flushleft}}

\date{}


\newcommand \printsplash {%
\noindent
  { \color {gray} \rule[-1ex]{0.5em}{7ex}\hspace{2em} }
  \begin{minipage}[b]{\textwidth - 4.7em}
    \onehalfspacing
    \emph{feed} --- Something supplied continuously.
    
    \emph{lot} --- One or more items auctioned or sold as a unit, separate from other items.
  \end{minipage}
  { \color {gray} \rule[-1ex]{0.2em}{7ex}\hspace{2em} }
  \vspace{4ex}
}

%%% END OF PREAMBLE %%%%%%%%%%%%%%%%%%%%%%%%%

\begin{document}
\maketitle
\printsplash

\paragraph{Purpose of this document}

This article is a feasibility study for a class of automated market makers (AMMs) that passively provide liquidity at a price based on the uniform clearing price (UCP) of a competitive batch auction.

It is not a whitepaper, specification, or proposal, and it makes no claim to exhaustiveness or completeness on any of the topics discussed --- particularly security. 
%
Would-be implementers of a feedlot AMM are advised to pursue thorough further investigations of these issues in the context into which they hope to deploy.

\newpage
\section{Introduction}

\section{General design principles}

Principles of feedlot:
\begin{itemize}
  \item
    Feedlot AMM LPs should enjoy cheap portfolio management, some yield, and protection from adverse selection.
    
  \item
    Feedlot AMM traders should enjoy low, predictable fees, control over execution time, and favourable prices at least for trades in the `correct' direction.
    
  \item
    It should not be economical to manipulate the UCP of the COW batch auction in order to trade on feedlot.
    
    Moreover, trading on a feedlot AMM should not make it too cheap to manipulate the COW UCP for any reason.

  \item 
    Although it is often possible to fall back on social consensus via CoW DAO slashing --- at least as long as misbehaviour is detectable --- wherever possible, feedlot should use incentive-compatible mechanisms to ensure correct operation in most cases.
    
  
\end{itemize}


\section{Architecture}

\section{Cryptoeconomic security}

The following inequality must be satisfied:
\[ \Delta_p \cdot\Vol_\mathrm{COW} > \Delta_p\cdot \Vol_\mathrm{feedlot} + P_\mathrm{ext}(\Delta_p)  \]
where:
\begin{itemize}
  \item $\Delta_p$ is the difference between the fair market price and the manipulated price.
  \item $\Vol_\mathrm{COW}$, resp.~$\Vol_\mathrm{feedlot}$ is the total volume on COW, resp.~Feedlot, for the pair.
  \item $P_\mathrm{ext}$ is the profit to be made from manipulating the price by $\Delta_p$ on external markets.
\end{itemize}

It is not feasible to estimate $P_\mathrm{ext}(p)$ for all time, and the developers of COW protocol can not reasonably expect to have any control on how COW price feeds are used in offchain markets.
%
The best we can hope to do is to ensure that
\[
  \Vol_\mathrm{COW} \gg \Vol_\mathrm{feedlot}
\]
and that it is difficult to guarantee atomic execution on both the COW batch and feedlot markets in opposite directions.

\section{The inventory problem}

Apart from trading fees and price updates, Uniswap or Balancer arbitrageurs provide a kind of portfolio management service to LPs.
%
This approach to portfolio management may be highly efficient, but it suffers from some problems:
\begin{itemize}
  \item 
    There is no room for competition on price to deliver this service: the cost to LPs is fixed in terms of the CFMM invariant, the initial reserves, and the price change.
    %
    Arbitrageurs compete to rebalance reserves on the basis of speed or bribes; the proceeds of this competition go to infrastructure providers rather than LPs.
    
  \item
    Portfolio management and price updates are bundled together in the same operation. There is no scope for these services to be priced or managed separately.
    
\end{itemize}

So far, we have not discussed any analogous mechanism for feedlot LPs.

\section{Conclusion}

This was a success.

\printbibliography
\end{document}
